\documentclass{article}
\usepackage{cite}
\usepackage{graphicx} % Required for inserting images

\title{Homework 2: A Paradox, Induction, and Research Basics}
\author{Casper Kristiansson}
\date{\today}

\begin{document}

\maketitle

\section{Achilles and the Tortoise}
The Achilles and the Tortoise paradox is part of one of Zeno's paradoxes \cite{black1951achilles}.  The paradox is built on a race between Achilles and a tortoise. The Tortoise gets a head start because it moves much slower than the Achilles. The paradox argues that every time Achilles catches up to a point where the tortoise is it will have moved on but not as far this time. Because of this, there will be an infinite amount of points that Achilles must reach to overtake the tortoise. Because of this Achilles can never surpass the tortoise. But in reality, we know that if Achilles is faster it will at some point in time surpass the tortoise. 

\begin{enumerate}
    \item Premise:
    \begin{enumerate}
        \item A Achilles races against a tortoise
        \item The tortoise gets a head start, for example, 100 meters
        \item Achilles moves faster than the tortoise
    \end{enumerate}
    \item Argument:
    \begin{enumerate}
        \item When Achilles reaches a point where the tortoise is the tortoise will have moved even further away.
        \item When Achilles reaches the next point it has moved further away which means there are infinite amounts of points that Achilles has tor each in order to pass the tortoise.
    \end{enumerate}
    \item Conclusion:
    \begin{enumerate}
        \item Given the following race Achilles will not surpass the tortoise
    \end{enumerate}
\end{enumerate}

The problem with this conclusion drawn from the paradox is regarding the infinite amounts of points that Achilles has to take to overpass the tortoise. While using limits we can see that there are an infinite amount of distances that Achilles has to take to overtake the tortoise will eventually converge to a finite distance. Meaning by applying what we know about math we can see that a sum of an infinite number of elements can sum to a finite number.

\section{When does induction work?}
The raven paradox is a problem or statement that involves the observation of a specific situation \cite{aronson1989bayesians}. The raven paradox is relevant to observing ravens. In an instance of generalization like saying that all ravens are black we would want to express our hypothesis that we are observing/trying to find nonblack ravens. But with this logical expression, we would also observe an apple or a table to be a non-black raven. 

\subsection{Raven Paradox Explanation}
There are different approaches or resolutions to this paradox. One is the resolution of relevance where you only consider evidence that is relevant to support the hypothesis. Another approach is the Bayesian solution \cite{aronson1989bayesians}. This solution uses a kind of point system to how much a piece of information supports the hypothesis. The point system is used to measure how much the evidence supports the hypothesis. Using these types of systems will help to when drawing any kind of decision based on an observation made and will improve the conclusion drawn from it.

\subsection{Induction}
The Goodman's paradox \cite{stemmer1975goodman} is a result of an experiment that Nelson Goodman performed. The experiment was about illustrating how challenging it can be to differentiate between rules that seem to always apply and occurrences that seem to just happen randomly.

A good example of when induction is wrong is the sunrise problem \cite{lee2020resolving}. The problem builds on the foundation that the sun rises every morning in a specific place like the East and then sets in the West. Using induction which uses the historical events to predict the future. But as we know the past doesn't always predict the future. In this problem as we know the Earth rotates which results in that the sun doesn't always rise in the east and sets in the west. What conclusion can be drawn from this is that if any type of observing patterns changes could lead to that the induction reasoning might be invalid. For example with the sunrise problem.

\section{Research Basics}
In Computer Science, there are two types of reasoning: inductive and deductive. Moreover, various philosophical perspectives such as positivism, realism, postmodernism, and critical realism are also acknowledged \cite{Walliman_2022}.

Deductive reasoning is especially popular when analyzing the performance and correctness of algorithms. The general approach with deductive reasoning and algorithm analysis has three different steps which include general rules/principles, specific cases, and conclusion. The first step is involved establishing the principles and theories of the algorithm which are general rules for it. The rules are then applied to a specific algorithm for example in an instance where an algorithm claims to sort a list it is then tested that it actually can sort a list. After applying the set of rules a conclusion can be derived about the correctness and performance of the algorithm.

An example of relativism/interpretivism in Computer science could be Cloud Computing adoption and the user experience with cloud products. In this type of area, a lot of research has been done to see how different businesses see and interact with different cloud services and if they choose to migrate to them. In many instances, an organization or business can either adopt or resist it. In a lot of these type of cases, there are a lot of values and factor that gets counted in like context experience, etc. This means that whenever an organization adopts cloud computing is not universally applicable. However, by understanding the reasoning of the organization or business that chooses to reject the adoption of cloud computing cloud providers can develop strategies for attracting these businesses by addressing their concerns. This also applies to the user interactions of the different cloud products.



\bibliographystyle{IEEEtran}
\bibliography{main}

\end{document}
