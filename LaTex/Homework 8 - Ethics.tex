\documentclass{article}
\usepackage{cite}
\usepackage{tabularx}
\usepackage{graphicx} % Required for inserting images
\usepackage{dirtytalk}
\usepackage{pgfplotstable} 
\usepackage{pgfplots}
\usepackage{datatool}
\usepackage{siunitx}
\usepackage[hyphens]{url}  % Allows line breaks at hyphens
\usepackage{hyperref}
\usepackage{dirtytalk}
\usepackage{graphicx}
\usepackage{microtype}

\hypersetup{
    colorlinks=true,
    linkcolor=blue,
    filecolor=blue,      
    urlcolor=blue,
    citecolor=blue,
}


\title{Homework 8 - Ethics\\HT2023}
\author{Casper Kristiansson}
\date{\today}

\begin{document}

\maketitle

\section{Research and Understanding}
I started off by reading the article "Open letter: we must stop killer robots before they are built" \cite{Openlett7:online} by Toby Walsh. The article introduces a call for a ban on the usage of autonomous weapons. Over 1000 AI and robotics researchers have signed an open letter to ban offensive autonomous weapons. The letter has been endorsed by notable figures such as Elon Musk and Stephen Hawking. The article is trying to get the backing from the United Nations to ban these weapons.

The second article I read was "Why You Shouldn’t Fear “Slaughterbots”" \cite{WhyYouSh92:online} by Paul Scharre. This article is critiquing a video called Slaughterbots. The article states \say{As a piece of propaganda, it works great. As a substantive argument for a ban on autonomous weapons, the video fails miserably.}. The article mentions that while autonomous drones like those sued by ISIS exist the video exaggerates. The video highlights that while certain parts of the technology exist most of the highlighted technology is not yet a reality.

\section{Should autonomous lethal weapons be banned, or subject to other forms of restrictions?}
The articles discuss a lot of important key points in both what is wrong with the usage of autonomous lethal weapons and arguments for the usage of it.

\subsection{Increased Efficiency and Effectiveness in Warfare}
\textbf{Argument:} Autonomous weapons can operate faster and more efficiently than human-operated weapons. This will lead to potentially reduced military casualties. \\

\noindent \textbf{Counterargument:} Increase in the efficiency in warfare will lead to more frequent usage of military force that escalates conflicts and could potentially increase the overall harm.

\subsection{Reduced Human Error and Emotion in Combat Decisions}
\textbf{Argument:} Machines don't suffer from fatigue, stress, or emotions. Because of this, it can perform better judgment which can lead to less error-prone decisions. \\

\noindent \textbf{Counterargument:} By removing human judgment and emotion could lead to inhuman warfare, which means that machines might not understand ethical and moral decisions in complex situations.

\subsection{Risk of Malfunction and Unpredictability}
\textbf{Argument:} Autonomous weapons could malfunction or be hacked which can lead to unintended or unauthorized use. This will pose a significant safety risk. \\

\noindent \textbf{Counterargument:} With technology advances lead to improved security and minimize the risk of malfunctions. Because of this, the risk of malfunctions or hacking can be lower than human-operated systems.

\section{Reflection and Moral Responsibilities in Engineering}
In the given scenario when a student is evaluating if they should do their thesis at a company that sells products to the military sector. When deciding whether to do this thesis it's important to take into account the personal opinion regarding the subject. Especially regarding the subject of companies working in the military sector, it's important to understand both the arguments and counterarguments for what is being done. Understanding the opinions of both sides can help the student to create their own opinion.

This means that from an ethical standpoint, it's important to understand what you think is right and not.  By working with a company that supplies products to the military sector, you have the opportunity to influence and ensure that these products uphold higher ethical standards. This is similar to the approach taken by the CEO of OpenAI, who wants to introduce regulations governing AI usage to ensure responsible and ethical implementation \cite{OpenAICE39:online}. 

\hspace{0cm}
\newpage

\bibliographystyle{acm}
\bibliography{main}

\end{document}
