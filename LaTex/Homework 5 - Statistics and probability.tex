\documentclass{article}
\usepackage{cite}
\usepackage{tabularx}
\usepackage{graphicx} % Required for inserting images
\usepackage{dirtytalk}
\usepackage{pgfplotstable} 
\usepackage{pgfplots}
\usepackage{datatool}
\usepackage{siunitx}
\usepackage[hyphens]{url}  % Allows line breaks at hyphens
\usepackage{hyperref}
\usepackage{dirtytalk}
\usepackage{graphicx}
\usepackage{microtype}

\hypersetup{
    colorlinks=true,
    linkcolor=blue,
    filecolor=blue,      
    urlcolor=blue,
    citecolor=blue,
}


\title{Homework 5: Statistics and probability}
\author{Casper Kristiansson}
\date{\today}

\begin{document}

\maketitle

\section{Probability}
\subsection{A}
We have three boxes, one with 2 gold coins, one with 1 gold coin and one silver coin, and lastly one with 2 silver coins. We know that the first coin we pick up is a gold coin. This means the box with 2 silver coins is not selected. This means that in the two remaining boxes, there was a 50\% chance of selecting the box with 2 gold coins and 50\% for selecting the one with 1 gold and 1 silver coin. This means there is a 50\% chance that the second coin is gold.

\subsection{B}
Bayes' theorem:

\begin{align*}
P(A|B) = \frac{P(B|A) \times P(A)}{P(B)}
\end{align*}

\begin{align*}
P(B) = P(B|A) \times P(A)+P(B| \neg A) \times P(\neg A)
\end{align*}


\noindent \(P(A)=\) Overall probability of breast cancer (0.3\%)

\noindent \(P(B)=\) Test result show positive

\noindent \(P(A|B)=\) Given that the result shows positive for breast cancer and the patient has breast cancer

\noindent We can start by calculating the P(B)

\begin{align*}
P(B) = P(B|A) \times P(A)+P(B| \neg A) \times P(\neg A) \\
P(B) = 0.8 \times 0.003 + 0.09 \times (1-0.003) \\
P(B) = 0.09213
\end{align*}

\noindent We then can use that to calculate the $P(A|H)$.

\begin{align*}
P(A|B) = \frac{P(B|A) \times P(A)}{P(B)} \\
P(A|B) = \frac{0.8 \times 0.003}{0.09213} \\
P(A|B) \approx 0.02605
\end{align*}

\noindent Meaning $\approx 2.6\%$ of the people that get a positive result from breast cancer actually have breast cancer.

\section{Confidence intervals and statistical significance}
\subsection{A}
We can first calculate the mean value of all values which is $9.869$. By using a confidence interval \cite{simundic2008confidence} with the most common level (95\%), we can calculate that the confidence interval is 9.586 to 10.152. The calculations are derived from the standard error of the mean.

\begin{align*}
CI= 9.869 \pm (0.95 \times SEM)
\end{align*}

\noindent Where SEM can be calculated by \cite{Standard4:online}:

\begin{align*}
SEM=\frac{\sigma}{\sqrt{n}}
\end{align*}

\noindent The actual calculations of the standard error of mean value were calculated using the library SciPy (scipy.stats.sem) \cite{scipysta19:online}.

\subsection{B}
To calculate the p-value we can use something called a two-sample t-test \cite{TwoSampl7:online}. A t-test is used to compare the averages of two groups and determine if differences between them are more likely to arise from random chance. To calculate the p-value we can use a library called SciPy (scipy.stats.ttest\_ind) \cite{scipysta44:online} that allows the input of the two data sets. The function gives us a p-value of $\approx 0.34$. Such a high value shows that there is no statistically significant difference between the two data sets' mean values. We know that values $\approx 0.05$ show that the null hypothesis is rejected.

\hspace{0cm}
\newpage

\bibliographystyle{acm}
\bibliography{main}

\end{document}