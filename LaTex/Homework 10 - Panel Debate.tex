\documentclass{article}
\usepackage{cite}
\usepackage{tabularx}
\usepackage{graphicx} % Required for inserting images
\usepackage{dirtytalk}
\usepackage{pgfplotstable} 
\usepackage{pgfplots}
\usepackage{datatool}
\usepackage{siunitx}
\usepackage[hyphens]{url}  % Allows line breaks at hyphens
\usepackage{hyperref}
\usepackage{dirtytalk}
\usepackage{graphicx}
\usepackage{microtype}

\hypersetup{
    colorlinks=true,
    linkcolor=blue,
    filecolor=blue,      
    urlcolor=blue,
    citecolor=blue,
}


\title{Homework 10 - Panel Debate\\HT2023}
\author{Casper Kristiansson}
\date{\today}

\begin{document}

\maketitle

\section{Is Computer Science Science?}
\subsection{Introduction}
Today's discussion will be centered around modern technology and academia especially regarding the subject "Is Computer Science Science?". The discussion aims to understand and explore how computer science aligns with the norms of scientific standards and what actually distinguishes it as a unique field of study compared to other science-related fields.

\subsection{Questions for the panel}
\begin{enumerate}
    \item In what way does computer science employ scientific methods that are similar to traditional sciences?
    \item Can software development processes be compared to experimental methods in traditional sciences? If so what similarities or differences are there?
\end{enumerate}

\subsection{Arguments and Counter Arguments}
\begin{enumerate}
    \item Engineering vs. Scientific Foundations
    \begin{enumerate}
        \item \textbf{Argument:} Computer science focuses more on creating technology and software which aligns more with engineering than pure science.
        \item \textbf{Counter Argument:} The creation of technology and software in computer science is based on principles and theories that align with physics. This means that in order to design and develop in computer science it will often consist of the steps of hypotheses, experimenting, and getting a result.
    \end{enumerate}
    \item Problem-solving and building solutions
     \begin{enumerate}
        \item \textbf{Argument:} Computer science is focused on problem-solving and building solutions which is more characteristic of technical engineering than pure science.
        \item \textbf{Counter Argument:} Problem-solving is a big part of many scientific experiments. The approach of computer science to solve complex problems using methods such as hypothesis, experiment, and analysis of results aligns with what the scientific process looks like.
    \end{enumerate}
\end{enumerate}

\section{Will generative AI transform software development?}
\subsection{Introduction}
Today's discussion will be centered around the rapid evolution of generative AI in the field of software development. This type of technology has enabled automation and enhanced the coding processes which has the possibility of revolutionizing the development of software. This discussion aims to understand the relationship between AI and software engineering and how AI can reshape coding and innovation in software development.

\subsection{Questions for the panel}
\begin{enumerate}
    \item How can generative AI change the educational and skill requirements for future software development roles?
    \item Can generative AI truly understand the requirements of complex software projects? Will AI replace software developers in the future?
\end{enumerate}

\subsection{Arguments and Counter Arguments}
\begin{enumerate}
    \item Will generative AI lead to unemployment among developers?
    \begin{enumerate}
        \item \textbf{Argument:} Generative AI, with its capabilities in automating coding tasks, could replace many functions currently performed by human developers. As these AI systems continue to grow they could be able to handle complex tasks which can reduce the demand of human programmers, especially in routine coding jobs.
        \item \textbf{Counter Argument:} Rather than leading to unemployment generative AI leads to productivity increase among developers. Meaning that it can free up time-heavy tasks allowing a developer to focus on more creative and complex problems.
    \end{enumerate}
    \item Should generative AI tools be allowed in programming assignments at KTH?
     \begin{enumerate}
        \item \textbf{Argument:} Allowing generative AI tools in education can enhance the learning experience, These tools can assist in understanding complex concepts thus helping students and improving their skills.
        \item \textbf{Counter Argument:} The usage of generative AI tools in programming assignments might worsen the learning process. Students might become too reliant on generative AI tools for coding solutions which will affect their fundamental knowledge.
    \end{enumerate}
    \item Can a user of generative AI tools accidentally violate intellectual property rights In programming or other applications), and will this hinder the adoption of generative models?
     \begin{enumerate}
        \item \textbf{Argument:} There is a risk in generative AI tools that they often learn based on copyrighted material. This means that their output might contain copyrighted material which can lead to potential legal risks for both the users and developers of these AI systems.
        \item \textbf{Counter Argument:} While there is a potential risk in the usage of intellectual property infringement in the learning of AI models, this problem can be solved with improved AI models that are better for creating and generating original content.
    \end{enumerate}
\end{enumerate}


\hspace{0cm}
\newpage

\bibliographystyle{acm}
\bibliography{main}

\end{document}