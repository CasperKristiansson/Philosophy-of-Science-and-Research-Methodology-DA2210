\documentclass{article}
\usepackage{cite}
\usepackage{graphicx} % Required for inserting images

\title{Homework 1: Popper and Truth}
\author{Casper Kristiansson}
\date{September 2023}

\begin{document}

\maketitle

\section{Science as Falsification}
One of the main parts of scientific exploration is the process of questioning and testing existing theories on how something works. Throughout this process hypotheses and theories are made that are under examination become either supported or falsified.

\subsection{Geocentric Model}
The Geocentric Model is an ancient scientific belief that Earth is placed at the center of the universe. The Geocentric Model is an example of a hypothesis or theory that has been proven to be false and has been falsified by other theories or hypotheses. This scientific belief was modeled around the cosmological model where all celestial bodies like the Sun, Moon, stars, and other planets revolve around the Earth. This scientific belief was especially popular in Western Civilizations and was also associated with the philosopher Aristotle \cite{Misc123}.

This belief that originated back to around 350 B.C. lasted for about 2000 years. In the year 1543 Nicolaus Copernicus proposed a new cosmological model where the model suggests that the planets including Earth revolve around the Sun \cite{geocentric_model_the_earth_centered}. After this theory was made multiple other theories contradicted the geocentric model and the reason for it being invalid. Another example of another law that showed that it was wrong was Newton's Law of Universal Gravitation. The law of motion and universal gravitation provided a framework for describing motion both on Earth and in Space.

Because more astronomical observations and the understanding of physics expanded resulted in the the Geocentric Model was insufficient and the cosmological model was replaced with the Heliocentric Model \cite{heliocentrism}.

\section{What is Truth}

\begin{enumerate}
	\item The program statement \textit{while (true) } gives an infinite loop.
 \begin{enumerate}
     \item \textbf{True}. A condition inside a while loop specifies how long the loop runs. In the current statement, the condition will always be true and the loop will run indefinitely.
     \item \textbf{Correspondence}. The given statement validates and shows how the logic of a programming language works. In the statement, the program will enter the loop and continue to execute it indefinitely.
 \end{enumerate}
	\item A quantum computer can find the prime factors of an integer in polynomial time.
 \begin{enumerate}
     \item \textbf{True}. Statement is correct and based on Shor's algorithm which is a quantum algorithm that can factor numbers in polynomial time \cite{factorization_2023}.
     \item \textbf{Coherence}. The statement is logically linked with other  statements (Shor's algorithm). Meaning that statement P1 is linked to statement P2.
 \end{enumerate}
	\item Apple's revenue increased in the third quarter compared to the same period last year.
 \begin{enumerate}
     \item \textbf{False}. This is false due to the nature that the third quarter of the current year hasn't passed yet making the statement false. 
     \item \textbf{Correspondence}. The statement is correspondence if the Apple revenue data was provided.
 \end{enumerate}
	\item Comments make it easier to modify programs.
 \begin{enumerate}
     \item \textbf{True}. Comments explaining what a program does make it easier to modify it and the program.
     \item \textbf{Intuitive}. This is because many other people believe that comments help to understand and modify a specific program.
 \end{enumerate}
	\item Agile development provides greater job satisfaction.
 \begin{enumerate}
     \item \textbf{Neither}. This links little to question 5 where some people think that comments don't help the understanding of a program. But In this case where Agile development is one of many different methodologies for work it cannot be true or false.
     \item \textbf{Intuitive}. Same as question 5 because many other people believe the statement is correct and or false other people agree.
 \end{enumerate}
	\item Vaccination prevented around 15 million deaths from COVID-19 globally from Dec 2020 to Dec 2021.
 \begin{enumerate}
     \item \textbf{True}. The statement is true due to being associated with the health organization data on COVID and how many lives vaccination prevented \cite{watson_global_2022}.
     \item \textbf{Correspondence}. If the data is provided and matches the statement because the data is correct the statement is correct.
 \end{enumerate}
	\item P is a proper subset of NP.
 \begin{enumerate}
     \item \textbf{True}. P is a proper subset of NP because "Because some \textit{NP-complete} problems are dependant upon SAT to produce a deterministic polynomial time solution for them, and because SAT does not have a deterministic polynomial time solution, then P is a proper subset of \textit{NP}. \(P\neq NP\)." states Jerrald Meek \cite{fal_2008}.
     \item \textbf{Correspondence}. The statement is correspondence true due to the statement has been proven to be true. Meaning that P is true which makes that this statement P is true.
 \end{enumerate}
	\item This statement is true.
 \begin{enumerate}
     \item \textbf{Neither}. The statement says it is true but doesn't reference anything specifying why it is true.
     \item \textbf{Intuitive}. The statement believes it is true which makes it true for the statement.
 \end{enumerate}
	\item This statement is false. (Note - interpret this as "Statement 9 in the second part of HW1 is false")
 \begin{enumerate}
     \item \textbf{Neither}. Because if the statement is false the actual statement is also false which means that it can also be true. If the statement is true it also becomes false in turn. Meaning that this turns into a paradoxical type of question.
     \item \textbf{Intuitive}.  Same as question 8, because statement P believes that it is false it is false.
 \end{enumerate}
\end{enumerate}


\bibliographystyle{IEEEtran}
\bibliography{main}

\end{document}
